\chapter{Conclusión}
A raíz de todo lo desarrollado en este proyecto, se puede concluir que efectivamente se cumplió el objetivo de diseñar y desarrollar un prototipo de robot de compañía móvil, potenciado por inteligencia artificial, capaz de interactuar de forma autónoma y asistir al usuario dentro de un entorno doméstico. Si bien el prototipo tiene limitaciones en cuanto a su movilidad y capacidades de interacción, se logró demostrar la viabilidad de integrar tecnologías accesibles como el \textbf{Arduino UNO} y el \textbf{Raspberry Pi 5} con modelos de lenguaje avanzados para crear un sistema funcional. La implementación de un modelo de lenguaje conversacional permitió que el robot respondiera preguntas y mantuviera conversaciones básicas.

\vspace{0.5cm}
Sin embargo, existen evidentes áreas de mejora, como la optimización del sistema de detección y evitación de obstáculos, una mejora en el modelo utilizado (entrenar un modelo especializado para el robot) y la incorporación de más sensores para una mejor percepción del entorno. Además, se recomienda explorar la posibilidad de integrar capacidades adicionales, como una pantalla para mostrar información visual (por ejemplo, emociones o respuestas gráficas), un diseño más robusto y estético para el chasis del robot y, por último, una mejora en el suministro de energía para el robot.

\vspace{0.5cm}
En resumen, el proyecto KubiBot demostró el potencial de la combinación entre robótica e inteligencia artificial, destacando las diversas posibilidades de mejora y desarrollo en futuros proyectos.