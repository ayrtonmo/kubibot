\chapter{Conclusión}
A raiz de todo lo desarrollado en este proyecto, se puede concluir que efectivamente se cumplio el objetivo de diseñar y desarrollar un prototipo de robot de compañia movil potenciado por inteligencia artificial, capaz de interactuar de forma autónoma y asistir al usuario dentro de un entorno doméstico. Si bien, el prototipo tiene limitaciones en cuanto a su movilidad y capacidades de interacción, se logró demostrar la viabilidad de integrar tecnologías accesibles como el \textbf{Arduino UNO} y el \textbf{Raspberry Pi 5} con modelos de lenguaje avanzados para crear un sistema funcional. La implementación de un modelo de lenguaje conversacional permitió que el robot respondiera preguntas y mantuviera conversaciones básicas/

\vspace{0.5cm}
Sin embargo, existen evidentes areas de mejoras, como lo pueden ser la optimizacion del sistema de deteccion y evitacion de obstaculos, una mejora en el modelo utilizado (entrenar un modelo especializado para el robot) y la incorporacion de mas sensores para una mejor percepcion del entorno. Ademas, se recomienda explorar la posibilidad de integrar capacidades adicionales, como una pantalla para mostrar inforamcion visual como emociones o respuestas graficas, un disenho mas robusto y estetico para el chasis del robot y por ultimo una mejora en la suministracion de poder para el robot.

\vspace{0.5cm}
En resumen, el proyecto KubiBot demostro el potencial de la combinacion entre robótica e inteligencia artificial, destacando las diversas posibilidades de mejora y desarrollo en futuros proyectos.