\chapter{Desarrollo}

\section{Análisis}
El punto de partida de este proyecto fue el interés en explorar las aplicaciones prácticas de la inteligencia artificial generativa en un formato físico, con el fin de crear un dispositivo más cercano al usuario.

El primer concepto consistía en utilizar un hardware disponible en la institución: una cabeza robótica estática. La idea era dotar a esta cabeza de capacidades conversacionales, aprovechando su estructura existente para simular interacción humana (movimiento de ojos o boca).

Sin embargo, tras un análisis preliminar, se descartó esta idea, y se prefirió crear un hardware de cero, ya que entregaría mayor libertad de diseño e integración. Esto llevó entonces hacía un concepto nuevo: un prototipo de robot móvil y compacto. Esta nueva dirección se eligió por dos ventajas estratégicas:

    \begin{enumerate}
        \item \textbf{Simplicidad de Integración:} Aunque la movilidad añade un desafío de navegación, diseñar un chasis propio desde cero simplifica enormemente la integración de los componentes seleccionados (Raspberry Pi, sensores, micrófono y altavoz) en un sistema cohesivo y funcional.
        
        \item \textbf{Mayor Atractivo e Interacción:} Se determinó que un robot capaz de moverse por la habitación y reaccionar físicamente a su entorno sería percibido por el público como un dispositivo más dinámico, útil y, en definitiva, más atractivo y cercano como "compañero", cumpliendo así el objetivo inicial del proyecto de una forma más efectiva.
    \end{enumerate}

    Con esto en mente, se realizó un análisis de los componentes necesarios para su funcionamiento. Estos se dividieron en dos módulos principales inicialmente independientes entre sí, para asegurar el buen funcionamiento de cada uno

    \begin{enumerate}
        \item \textbf{Movimiento:} Este módulo sería el responsable de toda la locomoción física del prototipo. Se compondría de un chasis estructural, un sistema de ruedas impulsadas por motores y un sensor ultrasónico para la detección de obstáculos. Para controlar el movimiento se escogió el microcontrolador Arduino Uno, debido a su simplicidad de programación y su disponibilidad en el departamento. 
        \item \textbf{Comunicación:} Modulo responsable de la comunicación con el usuario. Se determinó inicialmente el levantar localmente una IA en un Raspberry PI, debido a su diseño compacto y alta potencia. En ella además se conectarían los dispositivos de entrada y salida. Se determinó que la forma de comunicación más cercana y amigable sería a voz.
    \end{enumerate}

    Una vez la funcionalidad determinada, se analizó la presentación visual del prototipo. Por las características de los componentes y los recursos limitados, se necesitaba una carcasa ligera y a su misma vez accesible. Se optó entonces por crear o buscar un modelo para imprimir en 3D, lo que entonces implicó que el diseño además debiese ser simple para evitar problemas con el filamento o de ensamblaje.
    
    Finalmente se optó por una carcasa completamente cúbica, ya que no solo resulta sumamente fácil de ensamblar, sino que además le da al robot una apariencia suave. Este diseño entonces se bosquejó, para luego ser trabajado y adaptado por Nicolás Poblete.

\section{Movimiento}
