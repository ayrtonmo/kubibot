\chapter{Introducción}

La unión entre la robótica y la inteligencia artificial (IA) ha estado en auge en los últimos años. Si bien no es algo nuevo, la integración de estas dos disciplinas ha cobrado una relevancia significativa sobre todo con el avance exponencial que ha tenido la IA en lo que es el campo generativo y de procesamiento de lenguaje. Esto ha vuelto más accesible la incorporación de IA en sistemas robóticos tanto económicamente como técnicamente, permitiendo que estos sean más inteligentes y capaces de interactuar de manera más natural con los humanos.

\vspace{0.5cm}

En el presente proyecto, llamado `KubiBot', se busca desarrollar un prototipo de robot de compañia movil potenciado por IA, que sea capaz de interactuar conversacionalmente con los usuarios, respondiendo preguntas, proporcionando información, etc. Se pretende utilizar tecnologías modernas, pero relativamente accesibles para el departamento de ingeniería en computación e informática de la Universidad de Magallanes, con el fin de explorar las capacidades que puede ofrecer la inteligencia artificial en la robótica.

\vspace{0.5cm}

En este informe técnico se detallarán los objetivos del proyecto, el marco teórico, las tecnologías utilizadas, el proceso de desarrollo y los resultados obtenidos. Además, se discutirán posibles mejoras y futuras líneas de trabajo para continuar explorando esta unión entre la robótica y la inteligencia artificial.