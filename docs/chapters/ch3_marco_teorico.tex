\chapter{Marco teórico}

\section{Robótica móvil}

\section{Inteligencia artificial}

\section{Arduino UNO}
El Arduino UNO es una placa de desarrollo basada en el microcontrolador ATmega328P, ampliamente utilizada en proyectos de electrónica y robótica debido a su facilidad de uso y versatilidad. Cuenta con 14 pines digitales de entrada/salida, 6 entradas analógicas, un cristal oscilador de 16 MHz, una conexión USB, un conector de alimentación y un botón de reinicio.

El Arduino UNO permite a los usuarios programar y controlar dispositivos electrónicos mediante el entorno de desarrollo integrado (IDE) de Arduino, que utiliza un lenguaje de programación basado en C/C++.



\section{Raspberry PI 5}