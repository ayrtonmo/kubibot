% Fonts, layout, links
\usepackage[margin=1in]{geometry}
\usepackage{setspace}
\onehalfspacing
\usepackage{lmodern}
\usepackage[T1]{fontenc}
\usepackage[utf8]{inputenc}
\usepackage[spanish]{babel}
\usepackage[hidelinks]{hyperref}
% \hypersetup{
%   pdftitle={\ThesisTitle},
%   pdfauthor={\AuthorName}
% }

% Figures, tables, math
\usepackage{graphicx}
\usepackage{caption}
\usepackage{subcaption}
\usepackage{booktabs}
\usepackage{array}
\usepackage{longtable}
\usepackage{tabularx}
\usepackage{amsmath}
\usepackage{siunitx}
\usepackage{float}


% Chapter/section formatting
\usepackage{titlesec}
\titleformat{\chapter}[block]
  {\normalfont\bfseries\Large\centering} % formato del título
  {}{0pt}{} % etiqueta vacía: no muestra "CAPÍTULO N:"
\titlespacing*{\chapter}{0pt}{-0.5em}{1.5em}
\setcounter{secnumdepth}{3}
\setcounter{tocdepth}{3}

% Number figures/tables by chapter
\usepackage{chngcntr}
\counterwithin{figure}{chapter}
\counterwithin{table}{chapter}

% Include LOF/LOT/References in ToC
\usepackage[nottoc]{tocbibind}
\usepackage{tikz}
\usetikzlibrary{arrows.meta,positioning,shapes,calc,fit}
\usepackage{csquotes}
% Citation setup (edit style if needed)
\usepackage[backend=biber,style=ieee,sorting=none,maxnames=6]{biblatex}
\addbibresource{bib/references.bib}

% Handy macros for consistent names
\newcommand{\UnivLine}{\textbf{\University}}

% Para código
\usepackage[T1]{fontenc}
\usepackage[varqu]{zi4}
\usepackage{listings}
\usepackage{xcolor}

% Colores personalizados para código
\definecolor{codebg}{HTML}{F7F7F7}
\definecolor{codeframe}{HTML}{CCCCCC}
\definecolor{codekeyword}{HTML}{005CC5}
\definecolor{codestring}{HTML}{008000}
\definecolor{codecomment}{HTML}{6A737D}
\definecolor{codelinenumbers}{HTML}{999999}

% Estilo global de listings (pseudocódigo / código fuente)
\lstset{
  basicstyle=\ttfamily\small,
  backgroundcolor=\color{codebg},
  frame=single,
  rulecolor=\color{codeframe},
  frameround=tttt,
  numbers=left,
  numberstyle=\tiny\color{codelinenumbers},
  stepnumber=1,
  numbersep=8pt,
  keywordstyle=\bfseries\color{codekeyword},
  stringstyle=\color{codestring},
  commentstyle=\itshape\color{codecomment},
  showstringspaces=false,
  tabsize=4,
  breaklines=true,
  breakatwhitespace=true,
  columns=fullflexible,
  keepspaces=true,
  captionpos=b,
  xleftmargin=1.5em,
  framexleftmargin=1.5em,
  framexrightmargin=0.5em,
  framextopmargin=0.5em,
  framexbottommargin=0.5em
}

% Lenguaje y estilo para pseudocódigo con palabras clave coloreadas
\lstdefinelanguage{pseudocodigo}{%
  morekeywords={while,if,else,elif,for,return,break,continue,case,switch,True,False,AND,OR,NOT},%
  sensitive=false,
  morecomment=[l]{//},
  morecomment=[s]{/*}{*/},
  morestring=[b]
}

\lstdefinestyle{pseudocodigo}{%
  language=pseudocodigo,
  keywordstyle=\bfseries\color{codekeyword},
  keywordstyle={[2]\bfseries\color{codekeyword}},
  commentstyle=\itshape\color{codecomment},
  stringstyle=\color{codestring}
}

% Nombre en español para los listados
\renewcommand{\lstlistingname}{Listado}